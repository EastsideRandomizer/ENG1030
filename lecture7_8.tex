\documentclass[12pt]{article}
\usepackage{amsmath}
\begin{document}

\title{ENG1030 - Opreational Amplifiers }
\date{7 August, 2012}
\author{Gordon Ng}
\maketitle

\pagebreak
\tableofcontents
\pagebreak
\section{Learning Objectives}
\begin{itemize}
\item Understand what an amplifier does.
\end{itemize}
\section{What is an amplifier?}
\begin{itemize}
\item Makes things bigger (Mostly voltages, currents, electrical powers)
\item The outpit should replicate the input waveform.
\end{itemize}
\subsection{How It Powers}
Due to conservation, the energy needed to amplify the signal comes from power supply( also known as \textbf{power rail}.)\\
The voltage are usualy fixed.
\subsection{Single and Dual-Rail Supplies}
\subsubsection{Dual Rail Supply}
They require two supply connected in series but they are great for high power applications.
\subsubsection{Single Rail Supply}
They can burn out speakers.
\section{Amplifier's Bandwidth}
Bandwidth roughly translates to how rapidly its out put can change.
A 20-khz audio amplier's output can change its voltage from positive to negative and back roughly 20,000 times per second.
\pagebreak
\section{Opreational Amplifiers}
It contains many integrated circuits.\\
It is the basis for many low-signal circuits.

\begin{itemize}
\item  Current is sourced from the positive supply when need to make the output voltage higher.
\item Current is sunk to the negative supply when needed to make the output voltage
\item Output Voltage:
\begin{displaymath}
V_o=A_{OL}(v_1-v_2)
\end{displaymath}
\item The input voltage must be within the range of supply rail voltage.
\item A supply bypass capacitors provides current when the current needs it. \textbf{Always Use Them!} It fills up energy when we don't use it and empties when we use ot.
\item Gain:
\begin{displaymath}
Gain=\frac{V_o}{V_s}
\end{displaymath}
\begin{displaymath}
V_{IN}=v_S{R_i}{R_i+R_s}
\end{displaymath}
\begin{displaymath}
V_{out}=\frac{A_oV_{IN}}{R_o+R_L}
\end{displaymath}

\begin{displaymath}
\end{displaymath}
\pagebreak
\section{Unity Gain Buffer}
\subsection{Purpose}
\item It can stop signals flowing in the wrong direction.
\item Its \textbf{gain is independent of source and load resistances}
\item It has a gain of 1. $V_{Out}=IN_+=IN_-=V_{IN}$
\subsection{Simple Circuit}
\item We take a connection out of output and feed it back to $V_-$
\subsection{Non-ideal analysis II}
\begin{displaymath}
\frac{V_o}{V_s}=\frac{1}{1+\frac{R_o}{R_O+A_OR_I}}
\end{displaymath}
\subsection{Intutive Understanding}

\pagebreak
\section{Ideal Operational Amplifier}
\item No currents flows into or out of the input.(\textbf{infinite input impedance})
\item The op-amp's internal volrage Gain,$A_O$ is infinite. Therefore $IN_+=IN_-$
\item The op-amp's response to $IN_+$ and $IN_-$ changing the same amount in the same direction is zero. This is called \textbf{zero common mode voltage gain}.
\item The op-amp's output is connected to a perfect (Voltage-controlled) power ssource that has zero impedence.
\item The op-amp is infinitely fast to respond to any change in input(infinite bandwidth)
\item These assumptions are only valid where the output connects back to the \textbf{NEGATIVE} terminal
\pagebreak
\section{Basic Op-Amp Circuits}
\item Introduce Virtual Ground
\subsection{Block Diagram}
\item It whos what a designer might require of an op-amp circuit. 
\item It contains no details, only input and output details.
\subsection{Non-inverting Amplifier}
\item $V_{out} \succ IN_+=IN_-=V_{input}$ 
\item This system is still under Negative Feedback.
\item The value of $V_{out}$ is related to $R_1$ and $R_2$
\item $\frac{V_{out}}{V_{in}}=1+\frac{R_2}{R_1}$
\pagebreak

\subsubsection{Practical Implementation}
\item Should be close to the mean value of supply rails, which is normally 0V.
\item Choose the op-amp with regard to gain, bandwidth.....
\item output should be desgined to be always between the supply rail voltages.
\item Load resistance should be between $1k\Omega$ and $100k\Omega$
\item the ground connections should come to a common ground plane.
\item IC should be placed across the gap all the time.

\subsection{Inverting Amplifier}
\item Positive input terminal is connected to ground.
\item input signal is connected to the \textbf{negative} terminal.
\item gain=
\begin{displaymath}
\frac{V_o}{V_{in}}=-\frac{R_2}{R_1}
\end{displaymath}

\subsection{Summing Inverting Amplifier}

\begin{align}
v_o=-\frac{R}{R_{s1}}V_{s1}-\frac{R}{R_{S2}}V_{S2}
\end{align}

\pagebreak
\subsection{Differential Amplifier}
It has low input impedance.($R_{in}$ should be as large as possible.)
\begin{align}
V_o=\frac{R_2}{R_1}(v_{s2}-v_{s1})
\end{align}
Total impedence = $R_3 + R_1$
\subsubsection{Ground Loop}
As both point may not have a potential difference of zero. It might cause eror in received sensor voltage. \\
WIth a differenital amplifier the signal is now sent as a difference in voltage between the return wire and receive wire. \\
Twisted Pair reduces induced voltage due to time-varying magnetic field.
\subsection{Instrumental Amplifier}
This solves the problem of low input impedence as the unity gain buffer has very large imput impedance.

The Gain will be the product of the op-amp and the instrumental amplifier.
\subsection{Common Mode Voltage}
It acts as a noise which acts on $V_{s1}$ and $V_{S2}$
\end{itemize}
\end{document}
