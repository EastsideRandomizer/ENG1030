
\documentclass[12pt]{article}
\begin{document}
\section{Lecture 4}
\subsection{Lumped-Parameter Model}
\begin{itemize}
\item Interconnects have zero resistance
\item All point on a interconnect has same voltage.
\item Interconnects between elements are called nodes.
\item Two nodes are only considered as seperate nodes if there is element in between them.
\end{itemize}
\subsection{Kirchoff's Law}
The sum of the currents entering any node is zero.

This results in the fact that similar current charges repel each other.
\subsubsection{Note}
You just have to be consistent. Choose a reference direction and stick with it all the time. \\
$-I_x-10+5I_x+2=0$

\subsubsection{Example 2.5}
$
i_1-i_3-i_2=0 \\
i_1-i_4+i_6=0 \\
i_7+i_6-i_8=0 \\
$

\subsection{Loops and Branches}
\begin{itemize}
\item \textbf{Loop}: A loop is any closed path thtough in which is no node is encountered more than once.
\item \textbf{Branches}: Equal to number of elements.
\end{document}